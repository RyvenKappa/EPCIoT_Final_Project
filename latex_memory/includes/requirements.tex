\section{REQUIREMENTS ANALYSIS}
\subsection{Specifications required and implemented}

This section presents in the next tables the list of requirements needed to implement, as well as the implementation status for each specification.
\vspace{2\baselineskip}
\begin{table}[H]
    \begin{center}
        \begin{tabular}{p{0.1\textwidth} | p{0.7\textwidth} | p{0.1\textwidth}}
            Req. ID & Requirement description & Implemented\\
            \hline
            SR1 & The temperature in the range of -10ºC to 50ºC. Accuracy to one tenth of a degree. & Y\\
            \hline
            SR2 & The relative humidity in the range of 25\%HR to 75\%HR. Accuracy to one tenth of a percent. & Y\\
            \hline
            SR3 & Ambient light in \%, corresponding 0\% to total darkness and 100\% to maximum light. Accuracy to one tenth of a percent. & Y\\
            \hline
            SR4 & Soil moisture in \%, corresponding 0\% to total dryness and 100\% to maximum moisture. Accuracy to one tenth of a percent. & Y\\
            \hline
            SR5 & Colour of one leaf of the plant. The four associated parameters are clear, red, green, and blue values. & Y\\
            \hline
            SR6 & The global location of the plant should be registered. The GPS module also offers the current time (only time, the date is optional), that will be used to timestamp all the measurements taken by the system. & Y\\
            \hline
            SR7 & The acceleration of the plant. At least the three axes (X, Y and Z) values should be monitored. & Y\\
            \hline
            GR1 & The system must be robust and stable. & Y\\ %TODO ver que hago con esto
            \hline
            GR2 & Task partitioning and threads management should be stablished according to the requirements. & Y\\
            \hline
            GR3 & The system starts in TEST MODE and changes from one operating mode to the next one (TEST – NORMAL ADVANCED - TEST...) by pressing the blue button B1 on the B-L072Z-LRWAN1 board in a circular way. & Y\\
            \hline
        \end{tabular} 
    \end{center}
    \caption{General requirements implementation status}
    \label{ReqGeneral}
\end{table}

\vspace{2\baselineskip}

\begin{table}[H]
    \begin{center}
        \begin{tabular}{p{0.1\textwidth} | p{0.7\textwidth} | p{0.1\textwidth}}
            Req. ID & Requirement description & Implemented\\
            \hline
            TM1 & Check connections and sensor management. & Y\\
            \hline
            TM2 & All of the required variables should be monitored every 2 seconds. & Y\\
            \hline
            TM3 & The system sends every 2 seconds all the measured values to the computer (using the USB virtual COM port of the B-L072Z-LRWAN1 board). & Y\\
            \hline
            TM4 & The RGB LED should be coloured in the dominant colour detected by the colour sensor. & Y\\
            \hline
            TM5 & In this mode, the LED1 of the B-L072Z-LRWAN1 board should be ON. & Y\\
            \hline
        \end{tabular} 
    \end{center}
    \caption{Test mode requirements implementation status}
    \label{ReqTest}
\end{table}

\begin{table}[H]
    \begin{center}
        \begin{tabular}{p{0.1\textwidth} | p{0.7\textwidth} | p{0.1\textwidth}}
            Req. ID & Requirement description & Implemented\\
            \hline
            NM1 & All the required variables should be monitored with a cadence of 30s. & Y\\
            \hline
            NM2 & The system sends every 30 seconds all the measured values to the computer (using the USB virtual COM port of the B-L072Z-LRWAN1 board). & Y\\
            \hline
            NM3 & The system calculates the mean, maximum and minimum values of temperature, relative humidity, ambient light and soil moisture every hour. These values are sent to the computer when calculated. & Y\\
            \hline
            NM4 & The system calculates the dominant colour of the leave every hour. This means to calculate which colour has appeared as dominant more times during the last hour. This value is sent to the computer when calculated. & Y\\
            \hline
            NM5 & The system calculates the maximum and minimum values of the three axes (X, y and Z) of the accelerometer every hour. These values are sent to the computer when calculated. & Y\\
            \hline
            NM6 & The global location of the plant (coordinates) is sent to the computer every 30 seconds. This should include the GPS time (UTC) converted to local time. & Y\\
            \hline
            NM7 & Limits for every measured variable (temperature, humidity, ambient light, soil moisture, colour and acceleration) should be fixed. If the current values of the measured parameters are outside the limits, the RGB LED should indicate this situation using a different colour for every parameter. & Y\\
            \hline
            NM8 & In this mode, the LED2 of the B-L072Z-LRWAN1 board should be ON. & Y\\
            \hline
        \end{tabular} 
    \end{center}
    \caption{Normal mode requirements implementation status}
    \label{ReqNormal}
\end{table}
\subsection{Extra specifications implemented}
\begin{table}[H]
    \begin{center}
        \begin{tabular}{p{0.1\textwidth} | p{0.7\textwidth} | p{0.1\textwidth}}
            Req. ID & Requirement description & Implemented\\
            \hline
            E1 & The date is also extracted from the GPS data. & Y\\
            \hline
            E2 & Cosas & Y\\
            \hline
        \end{tabular} 
    \end{center}
    \caption{Extra requirements implemented}
    \label{ReqExtra}
\end{table}